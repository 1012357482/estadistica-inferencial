\PassOptionsToPackage{unicode=true}{hyperref} % options for packages loaded elsewhere
\PassOptionsToPackage{hyphens}{url}
%
\documentclass[]{article}
\usepackage{lmodern}
\usepackage{amssymb,amsmath}
\usepackage{ifxetex,ifluatex}
\usepackage{fixltx2e} % provides \textsubscript
\ifnum 0\ifxetex 1\fi\ifluatex 1\fi=0 % if pdftex
  \usepackage[T1]{fontenc}
  \usepackage[utf8]{inputenc}
  \usepackage{textcomp} % provides euro and other symbols
\else % if luatex or xelatex
  \usepackage{unicode-math}
  \defaultfontfeatures{Ligatures=TeX,Scale=MatchLowercase}
\fi
% use upquote if available, for straight quotes in verbatim environments
\IfFileExists{upquote.sty}{\usepackage{upquote}}{}
% use microtype if available
\IfFileExists{microtype.sty}{%
\usepackage[]{microtype}
\UseMicrotypeSet[protrusion]{basicmath} % disable protrusion for tt fonts
}{}
\IfFileExists{parskip.sty}{%
\usepackage{parskip}
}{% else
\setlength{\parindent}{0pt}
\setlength{\parskip}{6pt plus 2pt minus 1pt}
}
\usepackage{hyperref}
\hypersetup{
            pdftitle={Ejercicios Tema 2 - Estimación (continuación)},
            pdfauthor={Ricardo Alberich, Juan Gabriel Gomila y Arnau Mir},
            pdfborder={0 0 0},
            breaklinks=true}
\urlstyle{same}  % don't use monospace font for urls
\usepackage[margin=1in]{geometry}
\usepackage{color}
\usepackage{fancyvrb}
\newcommand{\VerbBar}{|}
\newcommand{\VERB}{\Verb[commandchars=\\\{\}]}
\DefineVerbatimEnvironment{Highlighting}{Verbatim}{commandchars=\\\{\}}
% Add ',fontsize=\small' for more characters per line
\usepackage{framed}
\definecolor{shadecolor}{RGB}{248,248,248}
\newenvironment{Shaded}{\begin{snugshade}}{\end{snugshade}}
\newcommand{\AlertTok}[1]{\textcolor[rgb]{0.94,0.16,0.16}{#1}}
\newcommand{\AnnotationTok}[1]{\textcolor[rgb]{0.56,0.35,0.01}{\textbf{\textit{#1}}}}
\newcommand{\AttributeTok}[1]{\textcolor[rgb]{0.77,0.63,0.00}{#1}}
\newcommand{\BaseNTok}[1]{\textcolor[rgb]{0.00,0.00,0.81}{#1}}
\newcommand{\BuiltInTok}[1]{#1}
\newcommand{\CharTok}[1]{\textcolor[rgb]{0.31,0.60,0.02}{#1}}
\newcommand{\CommentTok}[1]{\textcolor[rgb]{0.56,0.35,0.01}{\textit{#1}}}
\newcommand{\CommentVarTok}[1]{\textcolor[rgb]{0.56,0.35,0.01}{\textbf{\textit{#1}}}}
\newcommand{\ConstantTok}[1]{\textcolor[rgb]{0.00,0.00,0.00}{#1}}
\newcommand{\ControlFlowTok}[1]{\textcolor[rgb]{0.13,0.29,0.53}{\textbf{#1}}}
\newcommand{\DataTypeTok}[1]{\textcolor[rgb]{0.13,0.29,0.53}{#1}}
\newcommand{\DecValTok}[1]{\textcolor[rgb]{0.00,0.00,0.81}{#1}}
\newcommand{\DocumentationTok}[1]{\textcolor[rgb]{0.56,0.35,0.01}{\textbf{\textit{#1}}}}
\newcommand{\ErrorTok}[1]{\textcolor[rgb]{0.64,0.00,0.00}{\textbf{#1}}}
\newcommand{\ExtensionTok}[1]{#1}
\newcommand{\FloatTok}[1]{\textcolor[rgb]{0.00,0.00,0.81}{#1}}
\newcommand{\FunctionTok}[1]{\textcolor[rgb]{0.00,0.00,0.00}{#1}}
\newcommand{\ImportTok}[1]{#1}
\newcommand{\InformationTok}[1]{\textcolor[rgb]{0.56,0.35,0.01}{\textbf{\textit{#1}}}}
\newcommand{\KeywordTok}[1]{\textcolor[rgb]{0.13,0.29,0.53}{\textbf{#1}}}
\newcommand{\NormalTok}[1]{#1}
\newcommand{\OperatorTok}[1]{\textcolor[rgb]{0.81,0.36,0.00}{\textbf{#1}}}
\newcommand{\OtherTok}[1]{\textcolor[rgb]{0.56,0.35,0.01}{#1}}
\newcommand{\PreprocessorTok}[1]{\textcolor[rgb]{0.56,0.35,0.01}{\textit{#1}}}
\newcommand{\RegionMarkerTok}[1]{#1}
\newcommand{\SpecialCharTok}[1]{\textcolor[rgb]{0.00,0.00,0.00}{#1}}
\newcommand{\SpecialStringTok}[1]{\textcolor[rgb]{0.31,0.60,0.02}{#1}}
\newcommand{\StringTok}[1]{\textcolor[rgb]{0.31,0.60,0.02}{#1}}
\newcommand{\VariableTok}[1]{\textcolor[rgb]{0.00,0.00,0.00}{#1}}
\newcommand{\VerbatimStringTok}[1]{\textcolor[rgb]{0.31,0.60,0.02}{#1}}
\newcommand{\WarningTok}[1]{\textcolor[rgb]{0.56,0.35,0.01}{\textbf{\textit{#1}}}}
\usepackage{graphicx,grffile}
\makeatletter
\def\maxwidth{\ifdim\Gin@nat@width>\linewidth\linewidth\else\Gin@nat@width\fi}
\def\maxheight{\ifdim\Gin@nat@height>\textheight\textheight\else\Gin@nat@height\fi}
\makeatother
% Scale images if necessary, so that they will not overflow the page
% margins by default, and it is still possible to overwrite the defaults
% using explicit options in \includegraphics[width, height, ...]{}
\setkeys{Gin}{width=\maxwidth,height=\maxheight,keepaspectratio}
\setlength{\emergencystretch}{3em}  % prevent overfull lines
\providecommand{\tightlist}{%
  \setlength{\itemsep}{0pt}\setlength{\parskip}{0pt}}
\setcounter{secnumdepth}{0}
% Redefines (sub)paragraphs to behave more like sections
\ifx\paragraph\undefined\else
\let\oldparagraph\paragraph
\renewcommand{\paragraph}[1]{\oldparagraph{#1}\mbox{}}
\fi
\ifx\subparagraph\undefined\else
\let\oldsubparagraph\subparagraph
\renewcommand{\subparagraph}[1]{\oldsubparagraph{#1}\mbox{}}
\fi

% set default figure placement to htbp
\makeatletter
\def\fps@figure{htbp}
\makeatother


\title{Ejercicios Tema 2 - Estimación (continuación)}
\author{Ricardo Alberich, Juan Gabriel Gomila y Arnau Mir}
\date{Curso completo de estadística inferencial con R y Python}

\begin{document}
\maketitle

\hypertarget{estimaciuxf3n-continuaciuxf3n}{%
\section{Estimación
(continuación)}\label{estimaciuxf3n-continuaciuxf3n}}

\begin{enumerate}
\def\labelenumi{\arabic{enumi}.}
\tightlist
\item
  Supongamos que la cantidad de lluvia registrada en una cierta estación
  meteorológica en un día determinado está distribuida uniformemente en
  el intervalo \((0,b)\). Nos dan la siguiente muestra de los registros
  de los últimos 10 años en ese día: \[0,0,0.7,1,0.1,0,0.2,0.5,0,0.6\]
\end{enumerate}

\begin{quote}
Estimar el parámetro \(b\) a partir de su estimador \(\tilde b\).
\end{quote}

\begin{enumerate}
\def\labelenumi{\arabic{enumi}.}
\setcounter{enumi}{1}
\item
  Supongamos que el grado de crecimiento de unos pinos jóvenes en metros
  de altura en un año es una variable aleatoria normal con media y
  varianza desconocidas. Se registran los crecimientos de 5 árboles y
  los resultados son: \(0.9144,1.524,0.6096,0.4572\) y \(1.0668\)
  metros. Calcular los valores estimados de \(\mu\) y \(\sigma^2\) para
  esta muestra.
\item
  \(X\) es una variable geométrica con parámetro \(p\). Dada una muestra
  aleatoria de \(n\) observaciones de \(X\), cuál es el estimador de
  \(p\) por método de los momentos?
\item
  Se supone que el número de horas que funciona una bombilla LED es una
  variable exponencial con parámetro \(\lambda\). Dada una muestra de
  \(n\) duraciones, calcular el estimador por método de los momentos
  para \(\lambda\).
\item
  Si se supone que \(X\) esta distribuida uniformemente en el intervalo
  \((b-\frac14,b+5)\), ¿cuál es el estimador por método de los momentos
  para a \(b\) en base a una muestra aleatoria de \(n\) observaciones?
\item
  Supongamos que \(X\) es una variable aleatoria de Poisson con
  parámetro \(\lambda\). Dada una muestra aleatoria de \(n\)
  observaciones de \(X\), cuál es el estimador de máxima verosimilitud
  para a \(\lambda\)?
\item
  Supongamos que \(X\) está distribuida uniformemente en el intervalo
  \hbox{$\left(b-\frac12,b+\frac12\right)$.} ¿Cuál es el estimador de
  máxima verosimilitud para a \(b\) Dada una muestra aleatoria de tamaño
  \(n\) para \(X\)?
\item
  ¿Cuál es el estimador de máxima verosimilitud para al parámetro
  \(\lambda\) de una variable exponencial para a una muestra de tamaño
  \(n\)?
\item
  Se registran los tiempos de duración de 30 bombillas. Supongamos que
  el tiempo de duración de estas bombillas es una variable exponencial.
  Si la suma de los tiempos \(\sum x_i =32916\) horas, ¿cuál es el
  estimador de máxima verosimilitud para al parámetro de la distribución
  exponencial de duración de les bombillas?
\item
  Supongamos que \(X_1,X_2,\ldots,X_6\) es una muestra aleatoria de una
  variable aleatoria normal con media \(\mu\) y varianza \(\sigma^2\).
  Hallar la constante \(C\) tal que
  \[C\cdot\bigl({(X_1 -X_2)}^2 +{(X_3 -X_4)}^2 + 
   {(X_5 -X_6)}^2\bigr),\] sea un estimador sin sesgo de \(\sigma^2\).
\item
  Un vendedor de coches piensa que el número de ventas de coches nuevos
  que hace en un día no festivo\\
  es una variable aleatoria de Poisson con parámetro \(\lambda\).
  Examinando los registros del año anterior (que tuvo 310 días no
  festivos), observa que vendió un total de 279 coches. Calcular por
  método de la máxima verosimilitud la probabilidad que no venda ningún
  coche el próximo día de trabajo.
\item
  Supongamos que los años de vida de los hombres de los Estados Unidos
  Mexicanos están distribuidos normalmente con media \(\mu\) y varianza
  \(\sigma^2\). Una muestra aleatoria de \(n=10000\) antecedentes de
  mortalidad de hombres en México se obtuvo como resultado
  \(\overline{x}=72.1\) hombres , \textbackslash{}hbox\{\(s^2 =144\)
  años. Estimar por método de máxima verosimilitud la probabilidad que
  un hombre mexicano viva hasta los 50 años de edad y la probabilidad
  que un hombre no llegue a los 90 años.
\item
  Supongamos que \(\Theta_1\) y \(\Theta_2\) son estimadores sin sesgo
  de un parámetro desconocido \(\theta\), con varianzas conocidas
  \(\sigma_1^2\) y \(\sigma_2^2\), respectivamente. Se pide:
\end{enumerate}

\begin{quote}
a.) Demostrar que \(\Theta =(1-a)\cdot\Theta_1 +a\cdot \Theta_2\)
también es insesgado para cualquier valor de \(a\).
\end{quote}

\begin{quote}
b.) Hallar el valor de \(a\) que minimiza \(Var(\Theta)\).
\end{quote}

\begin{enumerate}
\def\labelenumi{\arabic{enumi}.}
\setcounter{enumi}{13}
\tightlist
\item
  Sea \(X\) una variable aleatoria \(N(\mu,\sigma)\) con \(\sigma\)
  conocida y \(X_1,\ldots, X_n\) una muestra aleatoria simple de \(X\).
  Consideremos los siguientes estimadores del parámetro
  \(\lambda =\mu^2\):
\end{enumerate}

\begin{eqnarray*}
T_1 & = & \overline{X}^2 = {\Biggl({\sum\limits_{i=1}^n
X_i\over n}\Biggr)}^2,\\  T_2 & = & {\sum\limits_{i=1}^n X_i^2 \over
n}-\sigma^2.
\end{eqnarray*}

\begin{quote}
Se pide:
\end{quote}

\begin{quote}
\begin{quote}
a.) Demostrar que \(T_1\) y \(T_2\) son estimadores consistentes.
\end{quote}
\end{quote}

\begin{quote}
\begin{quote}
b.) ¿Cuál estimador es más eficiente?
\end{quote}
\end{quote}

\begin{enumerate}
\def\labelenumi{\arabic{enumi}.}
\setcounter{enumi}{14}
\item
  Sea \(X_1,\ldots,X_{2n}\) una muestra aleatoria simple de una variable
  aleatoria \(N(\mu,\sigma)\). Sea: \[
  T=C\left({\left(\sum_{i=1}^{2n} X_i\right)}^2- 4 n\sum_{i=1}^{n}
  X_{2i} X_{2i-1}\right)
  \] un estimador del parámetro \(\sigma^2\). ¿Cuál es el valor de \(C\)
  para que \(T\) sea un estimador insesgado?
\item
  Una variable aleatoria \(X\) sigue la distribución de Rayleigh con
  parámetro \(\theta >0\) si es una variable aleatoria con valores
  \(x>0\) y función de densidad: \[
  f(x)=\frac{x}{\theta} e^{-\frac{x^2}{2\theta}}.
  \]
\end{enumerate}

\begin{quote}
Hallar estimadores del parámetro \(\theta\):
\end{quote}

\begin{quote}
\begin{quote}
a.) El método de los momentos.
\end{quote}
\end{quote}

\begin{quote}
\begin{quote}
b.) El método de la máxima verosimilitud.
\end{quote}
\end{quote}

\begin{enumerate}
\def\labelenumi{\arabic{enumi}.}
\setcounter{enumi}{16}
\tightlist
\item
  Consideremos una variable aleatoria \(X\) que es \(Exp(\lambda)\). Sea
  \(X_1,\ldots,X_n\) una muestra aleatoria simple de \(X\). Consideremos
  los siguientes estimadores del parámetro \(\frac{1}{\lambda^2}\):
\end{enumerate}

\begin{eqnarray*}
T_1 & = & \overline{X}^2,\\
T_2 & = & \frac{1}{2n}\sum\limits_{i=1}^n X_i^2.
\end{eqnarray*}

\begin{quote}
a.) ¿Son estimadores insesgados?
\end{quote}

\begin{quote}
b.) ¿Cuál de los dos estimadores es más eficiente? Indicación: La
variable aleatoria \(2\lambda \sum\limits_{i=1}^n X_i\) es una variable
\(\chi_{2n}^2\).
\end{quote}

\begin{enumerate}
\def\labelenumi{\arabic{enumi}.}
\setcounter{enumi}{17}
\item
  Sea \(X_1,\ldots,X_n\) una muestra aleatoria simple de una variable
  aleatoria \(X\) uniforme en el intervalo \([0,b]\). Consideremos el
  siguiente estimador del parámetro \(b\): \[
  \tilde{b}=a\sum\limits_{i=1}^n X_i,
  \] donde \(a\) es una constante. Calcular \(a\) para que \(\tilde{b}\)
  sea un estimador insesgado de \(b\), en este caso calcular
  \(Var(\tilde{b})\).
\item
  Hallar los estimadores por método de los momentos y por máxima
  verosimilitud del parámetro \(\alpha\) de la distribución de Maxwell:
  \[
  f(x)=\frac{4}{\alpha^3 \sqrt{\pi}} x^2 
  e^{-\frac{x^2}{\alpha^2}},\ x\geq 0,\ \alpha >0.
  \]
\item
  Sea \(X_1,\cdots,X_n\) una muestra aleatoria simple de una variable
  aleatoria \(X\) con \(E(X)=\mu\) y \(Var(X)=\sigma^2\). Consideremos
  los siguientes estimadores del parámetro \(\mu\): \[
  T_1 = \overline{X}=\frac{\sum\limits_{i=1}^n X_i}{n},\quad
  T_2 = k \sum_{i=1}^n y  X_i.
  \]
\end{enumerate}

\begin{quote}
Se pide:
\end{quote}

\begin{quote}
\begin{quote}
a.) El valor de la constante \(k\) para que \(T_2\) sea insesgado.
\end{quote}
\end{quote}

\begin{quote}
\begin{quote}
b.) Demostrar que \(T_1\) y \(T_2\) son consistentes.
\end{quote}
\end{quote}

\begin{quote}
\begin{quote}
c.) ¿Cuál estimador es más eficiente? Ayuda:
\(\sum\limits_{i=1}^n i^2 = \frac{n (n+1) (2n+1)}{6}\).
\end{quote}
\end{quote}

\begin{enumerate}
\def\labelenumi{\arabic{enumi}.}
\setcounter{enumi}{20}
\tightlist
\item
  Sea \(X_1,\ldots,X_{n}\) una muestra aleatoria simple de una variable
  aleatoria \(X\) tal que \(F_X\) depende de un parámetro desconocido
  \(\lambda\) con \(E(X)=\lambda\) y \(Var(X)=\lambda^2\). Consideremos
  el siguiente estimador de \(\lambda\): \[
  \tilde{\lambda}=
  \frac{1}{2}\left(\frac{1}{m}(X_1+\cdots X_m)+\frac{1}{n-m}(X_{m+1}+
  \cdots X_n)\right),\]
\end{enumerate}

con \(m=\frac{n}{3}\), en el que suponemos que \(n\) es múltiplo de
\(3\). Hallar la varianza de \(\tilde{\lambda}\).

\begin{enumerate}
\def\labelenumi{\arabic{enumi}.}
\setcounter{enumi}{21}
\tightlist
\item
  Sea \(X\) una variable aleatoria tal que \(E(X)=\mu\) y
  \(Var(X)=\sigma^2\). Sea \(X_1,X_2\) una muestra aleatoria simple de
  \(X\) de tamaño \(2\). Consideremos el siguiente estimador del
  parámetro \(\mu\): \(\tilde{\mu}=2 a X_1 + (1-2 a) X_2\). Hallar el
  valor de \(a\) que hace que el estadístico\\
  \(\tilde{\mu}\) sea el más eficiente posible.
\end{enumerate}

\hypertarget{soluciones}{%
\section{Soluciones}\label{soluciones}}

\begin{enumerate}
\def\labelenumi{\arabic{enumi}.}
\tightlist
\item
  Cargemos los datos en R
\end{enumerate}

\begin{Shaded}
\begin{Highlighting}[]
\NormalTok{muestra_lluvia=}\KeywordTok{c}\NormalTok{(}\DecValTok{0}\NormalTok{,}\DecValTok{0}\NormalTok{,}\FloatTok{0.7}\NormalTok{,}\DecValTok{1}\NormalTok{,}\FloatTok{0.1}\NormalTok{,}\DecValTok{0}\NormalTok{,}\FloatTok{0.2}\NormalTok{,}\FloatTok{0.5}\NormalTok{,}\DecValTok{0}\NormalTok{,}\FloatTok{0.6}\NormalTok{)}
\end{Highlighting}
\end{Shaded}

Nos dicen que los datos de la muestra provienen de una población modela
da pir una v.a. \(X\) con distribución \(U(0,b)\). Entonces
\(E(X)=\frac{1}{b-0}=\frac1b.\) por el método de los momentos estimamos
\(E(X)\) por \(\overline{X}\) luego \(\frac{1b}=E(X)\) de donde
\(b=\frac{1}{E(X)}\) y por lo tanto un estimador del parámetro \(b\) es
\(\hat{b}=\frac{1}{\overline{X}}.\) En nuestro caso y con R

\begin{Shaded}
\begin{Highlighting}[]
\NormalTok{media_lluvia=}\KeywordTok{mean}\NormalTok{(muestra_lluvia)}
\NormalTok{media_lluvia}
\end{Highlighting}
\end{Shaded}

\begin{verbatim}
## [1] 0.31
\end{verbatim}

\begin{Shaded}
\begin{Highlighting}[]
\NormalTok{bhat=}\DecValTok{1}\OperatorTok{/}\NormalTok{media_lluvia}
\NormalTok{bhat}
\end{Highlighting}
\end{Shaded}

\begin{verbatim}
## [1] 3.225806
\end{verbatim}

\begin{enumerate}
\def\labelenumi{\arabic{enumi}.}
\setcounter{enumi}{1}
\tightlist
\item
  Tenemos que \(X=\) crecimiento en metros de un pino joven en un año
  sigue una ley \(N(\mu,\sigma)\) de parámetros desconocidos. Tenemos un
  muestra que cargamos con R
\end{enumerate}

\begin{Shaded}
\begin{Highlighting}[]
\NormalTok{muestra_pinos=}\KeywordTok{c}\NormalTok{(}\FloatTok{0.9144}\NormalTok{, }\FloatTok{1.524}\NormalTok{, }\FloatTok{0.6096}\NormalTok{, }\FloatTok{0.4572}\NormalTok{ ,}\FloatTok{1.0668}\NormalTok{)}
\KeywordTok{mean}\NormalTok{(muestra_pinos)}
\end{Highlighting}
\end{Shaded}

\begin{verbatim}
## [1] 0.9144
\end{verbatim}

\begin{Shaded}
\begin{Highlighting}[]
\KeywordTok{var}\NormalTok{(muestra_pinos)}
\end{Highlighting}
\end{Shaded}

\begin{verbatim}
## [1] 0.1741932
\end{verbatim}

\begin{Shaded}
\begin{Highlighting}[]
\NormalTok{n=}\KeywordTok{length}\NormalTok{(muestra_pinos)}
\NormalTok{n}
\end{Highlighting}
\end{Shaded}

\begin{verbatim}
## [1] 5
\end{verbatim}

\begin{Shaded}
\begin{Highlighting}[]
\NormalTok{media_pinos=}\KeywordTok{sum}\NormalTok{(muestra_pinos)}\OperatorTok{/}\NormalTok{n}
\NormalTok{media_pinos}
\end{Highlighting}
\end{Shaded}

\begin{verbatim}
## [1] 0.9144
\end{verbatim}

\begin{Shaded}
\begin{Highlighting}[]
\NormalTok{varianza_pinos=}\KeywordTok{sum}\NormalTok{(muestra_pinos}\OperatorTok{^}\DecValTok{2}\NormalTok{)}\OperatorTok{/}\NormalTok{n}\OperatorTok{-}\NormalTok{media_pinos}\OperatorTok{^}\DecValTok{2}
\NormalTok{varianza_pinos}
\end{Highlighting}
\end{Shaded}

\begin{verbatim}
## [1] 0.1393546
\end{verbatim}

\begin{enumerate}
\def\labelenumi{\arabic{enumi}.}
\setcounter{enumi}{2}
\item
  Si \(X\) una v.a. discreta con distribución \(Ge(p)\) con
  \(D_X=\{0,1,2,\ldots\}\) en este caso sabemos que
  \(E(X)=\frac{1}{p}\), como \(\overline{X}\) es un estimador de
  \(E(X)\) podemos operar y \(\hat{p}=\frac{1}{\overline{X}}\) es un
  estimador por el método de los momentos del parámetro \(p\).
\item
  Ahora \(X\) una \(Exp(\lambda)\). La solución es similar que el caso
  anterior (no en vano la exponencial es la versión contniua de la v.a.
  geométrica).
\end{enumerate}

Sabemos que \(E(X)=\frac{1}{\lambda}\) luego un estimador del parámetro
\(\lambda\) de una población exponencial es
\(\hat{\lambda}=\frac{1}{\overline{X}}.\)

\begin{enumerate}
\def\labelenumi{\arabic{enumi}.}
\setcounter{enumi}{4}
\item
  Ahora \(X\) es sigue una ley \(U(b-\frac{1}{4},b+5)\) entonces
  \(E(X)=\frac{b-\frac{1}{4}+b+5}{2}=\frac{2\cdot b+\frac{19}{4}}{2}=b+\frac{19}{8}\).
  Así \(\hat{b}=\overline{X}-\frac{19}{8}.\)
\item
  Si \(X\) es una variable \(Po(\lambda)\) y tenemos una m.a.s
  \(X_1,X_2,\ldots,X_n\) de esa v.a; así su función de probabilidad es
  \(P(X_i=x_i)=\frac{\lambda^{x_i}}{x_i!}\cdot \mathrm{e}^{-\lambda}\)
  si \$x\_i=0,1,2,\ldots. La distribución de la muestra es
\end{enumerate}

\[
\begin{array}{lll}
P\left(X_1=x_1,X_2=x_2,\ldots,X_n=x_n\right)&=& P(X_1=x_1)\cdot P(X_2=x_2)\cdot\ldots \cdot P(X_n=x_n)\\
&=&\frac{\lambda^{x_1}}{x_1!}\cdot \mathrm{e}^{-\lambda}\cdot \frac{\lambda^{x_2}}{x_2!}\cdot \mathrm{e}^{-\lambda}\cdot \ldots \frac{\lambda^{x_n}}{x_n!}\cdot \mathrm{e}^{-\lambda}\\
&=&\frac{\lambda^{\sum_{i=1}^n}x_i}{x_1!\cdot x_2!\cdot\ldots\cdot x_n!} \mathrm{e}^{-n\cdot \lambda}.
\end{array}
\]

Así la función de verosimilitud es

\[
L(\lambda| x_1,x_2\ldots,x_n)=\frac{\lambda^{\displaystyle\sum_{i=1}^n x_i}}{x_1!\cdot x_2!\cdot\ldots\cdot x_n!}\cdot e^{-n\cdot \lambda}
\]

Queremos encontrar el valor de \(\lambda\) que máximiza
\(L(\lambda| x_1,x_2,\ldots, x_n)\) es decir

\[{\arg\, \max}_{\lambda} L(\lambda| x_1,x_2,\ldots,x_n)\] tomando
logaritmos tenemos que

\[
\begin{array}{lll}
\ln\left(L(\lambda| x_1,x_2\ldots , x_n)\right) &=&\ln\left(\frac{\lambda^{\displaystyle\sum_{i=1}^n x_i}}{x_1!\cdot x_2!\cdot\ldots\cdot x_n!}+ e^{-n\cdot \lambda}\right)\\
&=& \ln\left(\frac{\lambda^{\displaystyle\sum_{i=1}^n x_i}}{x_1!\cdot x_2!\cdot\ldots\cdot x_n!}\right)-n\cdot \lambda\\ 
&=& \ln\left(\lambda^{\displaystyle\sum_{i=1}^n x_i}\right)-\ln\left(x_1!\cdot x_2!\cdot\ldots\cdot x_n!\right)-n\cdot \lambda\\
&=& \left(\sum_{i=1}^n x_i\right)\cdot \ln(\lambda)-\ln\left(x_1!\cdot x_2!\cdot\ldots\cdot x_n!\right)-n\cdot \lambda
\end{array}
\]

derivando respecto de \(\lambda\)

\[
\frac{\partial}{\partial \lambda }\ln\left(L(\lambda| x_1,x_2\ldots , x_n)\right)=
\frac{\partial}{\partial \lambda } \left(\left(\sum_{i=1}^n x_i\right)\cdot \ln(\lambda)-\ln\left(x_1!\cdot x_2!\cdot\ldots\cdot x_n!\right)-n\cdot \lambda\right)=\frac{\left(\sum_{i=1}^n x_i}{\lambda}-n.
\]

el máximo se alacanza para el \$\lambda que cumpla la ecuación

\(\frac{\left(\sum_{i=1}^n x_i}{\lambda}-n=0\) de donde
\(\lambda =\frac{\sum_{i=1}^n x_i}}{n}=\overline{x}\).

Luego el estimador máximo verosimil de \$\lambda es \(\overline{X}\).

\begin{enumerate}
\def\labelenumi{\arabic{enumi}.}
\setcounter{enumi}{6}
\tightlist
\item
  Si \(X\) es \(U\left(b-\frac12,b+\frac12\right)\) su densidad es
  \(f_X(x)=\left\{\begin{array}{ll} \frac{1}{6} & \mbox{ si } b-\frac12<x<b+\frac12\\ 0 & \mbox{ en otro caso } \end{array}\righgt.\)
  su función de verosimilitud es constante NO TIENE EMV.
\end{enumerate}

\end{document}
