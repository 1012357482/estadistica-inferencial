\PassOptionsToPackage{unicode=true}{hyperref} % options for packages loaded elsewhere
\PassOptionsToPackage{hyphens}{url}
\PassOptionsToPackage{dvipsnames,svgnames*,x11names*}{xcolor}
%
\documentclass[]{article}
\usepackage{lmodern}
\usepackage{amssymb,amsmath}
\usepackage{ifxetex,ifluatex}
\usepackage{fixltx2e} % provides \textsubscript
\ifnum 0\ifxetex 1\fi\ifluatex 1\fi=0 % if pdftex
  \usepackage[T1]{fontenc}
  \usepackage[utf8]{inputenc}
  \usepackage{textcomp} % provides euro and other symbols
\else % if luatex or xelatex
  \usepackage{unicode-math}
  \defaultfontfeatures{Ligatures=TeX,Scale=MatchLowercase}
\fi
% use upquote if available, for straight quotes in verbatim environments
\IfFileExists{upquote.sty}{\usepackage{upquote}}{}
% use microtype if available
\IfFileExists{microtype.sty}{%
\usepackage[]{microtype}
\UseMicrotypeSet[protrusion]{basicmath} % disable protrusion for tt fonts
}{}
\IfFileExists{parskip.sty}{%
\usepackage{parskip}
}{% else
\setlength{\parindent}{0pt}
\setlength{\parskip}{6pt plus 2pt minus 1pt}
}
\usepackage{xcolor}
\usepackage{hyperref}
\hypersetup{
            pdftitle={Ejercicios con SOLUCIONES Tema 2 - Estimación. Taller 1},
            pdfauthor={Ricardo Alberich, Juan Gabriel Gomila y Arnau Mir},
            colorlinks=true,
            linkcolor=red,
            filecolor=Maroon,
            citecolor=blue,
            urlcolor=blue,
            breaklinks=true}
\urlstyle{same}  % don't use monospace font for urls
\usepackage[margin=1in]{geometry}
\usepackage{color}
\usepackage{fancyvrb}
\newcommand{\VerbBar}{|}
\newcommand{\VERB}{\Verb[commandchars=\\\{\}]}
\DefineVerbatimEnvironment{Highlighting}{Verbatim}{commandchars=\\\{\}}
% Add ',fontsize=\small' for more characters per line
\usepackage{framed}
\definecolor{shadecolor}{RGB}{248,248,248}
\newenvironment{Shaded}{\begin{snugshade}}{\end{snugshade}}
\newcommand{\AlertTok}[1]{\textcolor[rgb]{0.94,0.16,0.16}{#1}}
\newcommand{\AnnotationTok}[1]{\textcolor[rgb]{0.56,0.35,0.01}{\textbf{\textit{#1}}}}
\newcommand{\AttributeTok}[1]{\textcolor[rgb]{0.77,0.63,0.00}{#1}}
\newcommand{\BaseNTok}[1]{\textcolor[rgb]{0.00,0.00,0.81}{#1}}
\newcommand{\BuiltInTok}[1]{#1}
\newcommand{\CharTok}[1]{\textcolor[rgb]{0.31,0.60,0.02}{#1}}
\newcommand{\CommentTok}[1]{\textcolor[rgb]{0.56,0.35,0.01}{\textit{#1}}}
\newcommand{\CommentVarTok}[1]{\textcolor[rgb]{0.56,0.35,0.01}{\textbf{\textit{#1}}}}
\newcommand{\ConstantTok}[1]{\textcolor[rgb]{0.00,0.00,0.00}{#1}}
\newcommand{\ControlFlowTok}[1]{\textcolor[rgb]{0.13,0.29,0.53}{\textbf{#1}}}
\newcommand{\DataTypeTok}[1]{\textcolor[rgb]{0.13,0.29,0.53}{#1}}
\newcommand{\DecValTok}[1]{\textcolor[rgb]{0.00,0.00,0.81}{#1}}
\newcommand{\DocumentationTok}[1]{\textcolor[rgb]{0.56,0.35,0.01}{\textbf{\textit{#1}}}}
\newcommand{\ErrorTok}[1]{\textcolor[rgb]{0.64,0.00,0.00}{\textbf{#1}}}
\newcommand{\ExtensionTok}[1]{#1}
\newcommand{\FloatTok}[1]{\textcolor[rgb]{0.00,0.00,0.81}{#1}}
\newcommand{\FunctionTok}[1]{\textcolor[rgb]{0.00,0.00,0.00}{#1}}
\newcommand{\ImportTok}[1]{#1}
\newcommand{\InformationTok}[1]{\textcolor[rgb]{0.56,0.35,0.01}{\textbf{\textit{#1}}}}
\newcommand{\KeywordTok}[1]{\textcolor[rgb]{0.13,0.29,0.53}{\textbf{#1}}}
\newcommand{\NormalTok}[1]{#1}
\newcommand{\OperatorTok}[1]{\textcolor[rgb]{0.81,0.36,0.00}{\textbf{#1}}}
\newcommand{\OtherTok}[1]{\textcolor[rgb]{0.56,0.35,0.01}{#1}}
\newcommand{\PreprocessorTok}[1]{\textcolor[rgb]{0.56,0.35,0.01}{\textit{#1}}}
\newcommand{\RegionMarkerTok}[1]{#1}
\newcommand{\SpecialCharTok}[1]{\textcolor[rgb]{0.00,0.00,0.00}{#1}}
\newcommand{\SpecialStringTok}[1]{\textcolor[rgb]{0.31,0.60,0.02}{#1}}
\newcommand{\StringTok}[1]{\textcolor[rgb]{0.31,0.60,0.02}{#1}}
\newcommand{\VariableTok}[1]{\textcolor[rgb]{0.00,0.00,0.00}{#1}}
\newcommand{\VerbatimStringTok}[1]{\textcolor[rgb]{0.31,0.60,0.02}{#1}}
\newcommand{\WarningTok}[1]{\textcolor[rgb]{0.56,0.35,0.01}{\textbf{\textit{#1}}}}
\usepackage{graphicx,grffile}
\makeatletter
\def\maxwidth{\ifdim\Gin@nat@width>\linewidth\linewidth\else\Gin@nat@width\fi}
\def\maxheight{\ifdim\Gin@nat@height>\textheight\textheight\else\Gin@nat@height\fi}
\makeatother
% Scale images if necessary, so that they will not overflow the page
% margins by default, and it is still possible to overwrite the defaults
% using explicit options in \includegraphics[width, height, ...]{}
\setkeys{Gin}{width=\maxwidth,height=\maxheight,keepaspectratio}
\setlength{\emergencystretch}{3em}  % prevent overfull lines
\providecommand{\tightlist}{%
  \setlength{\itemsep}{0pt}\setlength{\parskip}{0pt}}
\setcounter{secnumdepth}{5}
% Redefines (sub)paragraphs to behave more like sections
\ifx\paragraph\undefined\else
\let\oldparagraph\paragraph
\renewcommand{\paragraph}[1]{\oldparagraph{#1}\mbox{}}
\fi
\ifx\subparagraph\undefined\else
\let\oldsubparagraph\subparagraph
\renewcommand{\subparagraph}[1]{\oldsubparagraph{#1}\mbox{}}
\fi

% set default figure placement to htbp
\makeatletter
\def\fps@figure{htbp}
\makeatother

\renewcommand{\contentsname}{Contenidos}

\title{Ejercicios con SOLUCIONES Tema 2 - Estimación. Taller 1}
\author{Ricardo Alberich, Juan Gabriel Gomila y Arnau Mir}
\date{Curso completo de estadística inferencial con R y Python}

\begin{document}
\maketitle

{
\hypersetup{linkcolor=blue}
\setcounter{tocdepth}{2}
\tableofcontents
}
\hypertarget{estimaciuxf3n-taller-1}{%
\section{Estimación taller 1}\label{estimaciuxf3n-taller-1}}

\hypertarget{ejercicio-1}{%
\subsection{Ejercicio 1}\label{ejercicio-1}}

El fabricante SMART\_LED fabrica bombillas led inteligentes y de alta
gama. Supongamos que la vida de de estas bombillas sigue una
distribución exponencial de parámetro \(\lambda\). Si tomamos una
muestra aleatoria de tamaño \(n\) de estas bombillas y representamos por
\(X_i\) la duración de la \(i-\)ésima bombilla para \(i=1,\ldots,n\),
¿cuál es la función de densidad conjunta de la muestra?

\hypertarget{ejercicio-2}{%
\subsection{Ejercicio 2}\label{ejercicio-2}}

Sean \(X_1,X_2,\ldots,X_{10}\) variables aleatorias que son una muestra
aleatoria simple de una v.a. \(X\). a. Dividimos la muestra en dos
partes: de forma que la primera son los \(5\) primeros valores y la
segunda los restantes. ¿Son independientes las dos partes? b. Volvemos a
dividir la muestra en dos partes: la primera está formada por los \(5\)
valores más pequeños y la segunda por el resto. ¿Son independientes las
dos partes?

\hypertarget{ejercicio-3}{%
\subsection{Ejercicio 3}\label{ejercicio-3}}

Un fabricante de motores pone a prueba \(6\) motores sobre el mismo
prototipo de coche de competición. Para probar que los motores tienes
las mismas prestaciones se someten a distintas pruebas en un circuito.
Las velocidades máximas en 10 vueltas al circuito de cada motor tras la
prueba son \(190,195,193,177,201\) y \(187\) en Km/h. Estos valores
forman una muestra aleatoria simple de la variable
\(X=\mbox{velocidad máxima de un motor en 10 vueltas.}\) Se pide
calcular los valores observados de los siguientes estadísticos de la
muestra: a. \(\overline{X}\). b. \(\tilde{S}^2\). c. Mediana. d.
\(X_{(4)}\) (valor que ocupa el cuarto lugar ordenados los valores de
menor a mayor).

\hypertarget{ejercicio-4}{%
\subsection{Ejercicio 4}\label{ejercicio-4}}

¿Cuál es la probabilidad de que el máximo de de una muestra de tamaño
\(n=10\) de una v.a. uniforme en el intervalo \((0,1)\) sea mayor que
\(0.9\)? ¿Cuál es la probabilidad de sea menor que \(\frac12\)?

\hypertarget{ejercicio-5}{%
\subsection{Ejercicio 5}\label{ejercicio-5}}

Sea \(X_1,X_2,\ldots,X_n\) una muestra aleatoria simple de una variable
aleatoria normal de parámetros \(\mu\) y \(\sigma\). Denotemos por
\(X_{(1)}\leq X_{(2)}\leq ,\ldots,\leq X_{(n)}\) la muestra ordenada de
menor a mayor. a. Calcular la funciones de densidad del mínimo
\(X_{(1)}\) y del máximo \(X_{(n)}\) b. ¿Alguna de estas variables sigue
una distribución normal?

\hypertarget{ejercicio-6}{%
\subsection{Ejercicio 6}\label{ejercicio-6}}

Consideremos la muestra aleatoria simple \(X_1,X_2,\ldots,X_n\) de una
v.a \(X\) de media \(\mu\) y varianza \(\sigma^2\) desconocidas.
Definimos
\[\overline{X}=\frac1n \sum\limits_{i=1}^n X_i\mbox{ y } T=\frac{\sqrt{n}\cdot(\overline{X}-\mu)}{\sigma}.\]

\begin{enumerate}
\def\labelenumi{\alph{enumi}.}
\tightlist
\item
  ¿Cuál es la distribución de \(T\)?
\item
  ¿Es \(T\) un estadístico?
\end{enumerate}

\hypertarget{ejercicio-7}{%
\subsection{Ejercicio 7}\label{ejercicio-7}}

Consideremos la muestra aleatoria simple \(X_1,X_2,\ldots,X_n\) de
tamaño \(n=10\) de una v.a \(X\) normal estándar. Calculad
\(P\left(2.56<\sum\limits_{i=1}^{10} X_i^2 <18.31\right)\).

\hypertarget{ejercicio-8}{%
\subsection{Ejercicio 8}\label{ejercicio-8}}

Consideremos la muestra aleatoria simple \(X_1,X_2,\ldots,X_{n}\) de
tamaño \(n=10\) de una v.a \(X\) normal \(N(\mu=2,\sigma=4)\). Definimos
la siguiente variable aleatoria
\(Y=\frac{\sum\limits_{i=1}^{10}{(X_i-2)}^2}{16}\). Calculad
\(P(Y\leq 2.6)\)

\hypertarget{soluciones}{%
\section{Soluciones}\label{soluciones}}

\hypertarget{soluciuxf3n-ejercicio-1}{%
\subsection{Solución ejercicio 1}\label{soluciuxf3n-ejercicio-1}}

Cada \(X_i\) sigue una ley \(Exp(\lambda)\) la densidad es

\[
f_{X_i}(x_i)=\left\{
\begin{array}{ll}
\lambda\cdot\mathrm{e}^{-\lambda\cdot x_i} & \mbox{si }x_i>0\\
0 & \mbox{si }x_i\leq 0
\end{array}
\right.
\]

Así la densidad conjunta de la muestra es

\[\begin{array}{rl}
f(x_1,x_2,\ldots x_n) &=f_{X_1}(x_1)\cdot f_{X_2}(x_2)\cdot \ldots\cdot f_{X_n}(x_n)\\
&=
\left\{
\begin{array}{ll}
\lambda^n \cdot\mathrm{e}^{-\lambda\cdot \sum_{i=1}^n x_i} & \mbox{si }x_i>0 \mbox{ para todo } i=1,2,\ldots, n\\
0 & \mbox{si } x_i\leq 0 \mbox{ para algún } i=1,2,\ldots, n 
\end{array}
\right.
.
\end{array}
\]

\hypertarget{soluciuxf3n-ejercicio-2}{%
\subsection{Solución ejercicio 2}\label{soluciuxf3n-ejercicio-2}}

En el primer caso las muestras son independientes, saber los resultados
del los 5 primeros no aporta información sobre los 5 últimos. En el
segundo caso sí que aporta información pues los valores de la segunda
parte deben ser mayores que todos los de la primera parte, luego no son
independientes.

\hypertarget{soluciuxf3n-ejercicio-3}{%
\subsection{Solución ejercicio 3}\label{soluciuxf3n-ejercicio-3}}

Lo calcularemos con R

\begin{Shaded}
\begin{Highlighting}[]
\NormalTok{x=}\KeywordTok{c}\NormalTok{(}\DecValTok{190}\NormalTok{,}\DecValTok{195}\NormalTok{,}\DecValTok{193}\NormalTok{,}\DecValTok{177}\NormalTok{,}\DecValTok{201}\NormalTok{,}\DecValTok{187}\NormalTok{)}
\NormalTok{x}
\end{Highlighting}
\end{Shaded}

\begin{verbatim}
## [1] 190 195 193 177 201 187
\end{verbatim}

\begin{Shaded}
\begin{Highlighting}[]
\NormalTok{n=}\KeywordTok{length}\NormalTok{(x)}
\NormalTok{n }\CommentTok{# tamaño de la muestra}
\end{Highlighting}
\end{Shaded}

\begin{verbatim}
## [1] 6
\end{verbatim}

\begin{Shaded}
\begin{Highlighting}[]
\KeywordTok{mean}\NormalTok{(x)}\CommentTok{# media}
\end{Highlighting}
\end{Shaded}

\begin{verbatim}
## [1] 190.5
\end{verbatim}

\begin{Shaded}
\begin{Highlighting}[]
\KeywordTok{var}\NormalTok{(x)}\CommentTok{# variana muestral con la función var}
\end{Highlighting}
\end{Shaded}

\begin{verbatim}
## [1] 66.3
\end{verbatim}

\begin{Shaded}
\begin{Highlighting}[]
\KeywordTok{sum}\NormalTok{((x}\OperatorTok{-}\KeywordTok{mean}\NormalTok{(x))}\OperatorTok{^}\DecValTok{2}\NormalTok{)}\OperatorTok{/}\NormalTok{(n}\DecValTok{-1}\NormalTok{) }\CommentTok{# variana muestral calculada directamente con R}
\end{Highlighting}
\end{Shaded}

\begin{verbatim}
## [1] 66.3
\end{verbatim}

\begin{Shaded}
\begin{Highlighting}[]
\KeywordTok{median}\NormalTok{(x)}
\end{Highlighting}
\end{Shaded}

\begin{verbatim}
## [1] 191.5
\end{verbatim}

\begin{Shaded}
\begin{Highlighting}[]
\KeywordTok{sort}\NormalTok{(x) }\CommentTok{# muestra ordenada}
\end{Highlighting}
\end{Shaded}

\begin{verbatim}
## [1] 177 187 190 193 195 201
\end{verbatim}

\begin{Shaded}
\begin{Highlighting}[]
\KeywordTok{sort}\NormalTok{(x)[}\DecValTok{4}\NormalTok{] }\CommentTok{# M_(4) el cuarto valor de la muestra ordenada}
\end{Highlighting}
\end{Shaded}

\begin{verbatim}
## [1] 193
\end{verbatim}

\hypertarget{soluciuxf3n-ejercicio-4}{%
\subsection{Solución ejercicio 4}\label{soluciuxf3n-ejercicio-4}}

La primera probabilidad es
\(P(\max\{X_1,\ldots,X_n\}\geq 0.9)=1-P(\max\{X_1,\ldots,X_n\}\leq 0.9)=1-P(X_1\leq 0.9,X_2\leq 0.9,\ldots, X_n\leq 0.9)=1-P(X_1\leq 0.9)\cdot P(X_2\leq 0.9)\cdot\ldots\cdot P(X_n\leq 0.9)=1-0.9^10= 0.6513.\)

La segunda es
\(P(\max\{X_1,\ldots,X_n\}\leq 0.1)=P(X_1\leq 0.1,X_2\leq 0.1,\ldots, X_n\leq 0.1)=P(X_1\leq 0.1)\cdot P(X_2\leq 0.1)\cdot\ldots\cdot P(X_n\leq 0.1)=0.1^{10}= \ensuremath{10^{-10}}\)

Hemos utilizado que la distribución uniforme

\[
P(X_i\leq x)=
\left\{\begin{array}{ll}
0 & \mbox{ si } x\leq 0\\
x & \mbox{ si } 0< x < 1\\
1 & \mbox{ si }  x\geq 1
\end{array}
\right.
\]

\hypertarget{soluciuxf3n-ejercicio-5}{%
\subsection{Solución ejercicio 5}\label{soluciuxf3n-ejercicio-5}}

Sea \(F_X\) la distribución de la variable que se muestrea entonces
\(F_{X_i}=F_X\) para \(i=1,2,\ldots,n\).

La distribucion del máximo es

\[
\begin{array}{lr}
P(\max\{X_1,\ldots,X_n\}\leq x)&=P(X_1\leq x,X_2\leq x,\ldots, X_n\leq x)\\
&=P(X_1\leq x)\cdot P(X_2\leq x)\cdot\ldots\cdot P(X_n\leq x)\\
&=F_{X_1}(x)\cdot F_{X_1}(x)\ldots\cdot F_{X_n}(x)=F_{X}(x)^n
\end{array}.
\]

La distribucion del mínimo es

\(P(\min\{X_1,\ldots,X_n\}\leq x)=1-P(\min\{X_1,\ldots,X_n\}\geq x)= 1- P(X_1\geq x,X_2\leq x,\ldots, X_n\geq x)=1-\left(P(X_1\geq x)\cdot P(X_2\geq x)\cdot\ldots\cdot P(X_n\geq x)\right)=1-P(X_1\geq x)\cdot P(X_2\geq x)\cdot\ldots\cdot P(X_n\geq x)= 1-\left(1-P(X_1\leq x)\right) \cdot \left(1-P(X_1\leq x)\right)\cdot\ldots\cdot\left(1-P(X_1\leq x)\right)= 1-\left(1-F_{X}(x)\right)\cdot \left(1-F_{X}(x)\right)\ldots\cdot \left(1-F_{X}(x)\right)=1- (1-F_{X}(x))^n.\)

No son normales. Se deja como ejercicio derivar la función de
distribución del máximo (o del mínimo) para el caso \(n=2\) y comprobar
que no es una gaussiana.

\hypertarget{soluciuxf3n-ejercicio-6}{%
\subsection{Solución ejercicio 6}\label{soluciuxf3n-ejercicio-6}}

Ahora tenemos una muestra aleatoria simple de una distribución de media
\(\mu\) y desviación típica sigma y como siempre tenemos el estadśitico
\(\overline{X}\).

\begin{enumerate}
\def\labelenumi{\alph{enumi}.}
\tightlist
\item
  Nos piden la distribución de
  \(T=\frac{\sqrt{n}\cdot (\overline{X}-\mu)}{\sigma}\) operando
  \[T=\frac{\sqrt{n}\cdot(\overline{X}-\mu)}{\sigma}=
  \frac{\overline{X}-\mu}{\frac{\sigma}{\sqrt{n}}}\]
\end{enumerate}

Ahora sabemos que la distribución de \(T\) por el Teorema Central de
Límite converge en distribución a una normal estándar cuando
\(n\to\infty\).

Además si las variables fueran normales \(T\) seguirá distribución
normal estándar.

\begin{enumerate}
\def\labelenumi{\alph{enumi}.}
\setcounter{enumi}{1}
\tightlist
\item
  Claro que \(T\) es un estadístico, ya que estadístico es cualquier
  función de una muestra. Además si nos fijamos bienes simplemente la
  tipificación del estadístico \(\overline{X}\).
\end{enumerate}

\hypertarget{soluciuxf3n-ejercicio-7}{%
\subsection{Solución ejercicio 7}\label{soluciuxf3n-ejercicio-7}}

Como se una muestra de una normal estándar tenemos que \(\mu=0\) y
\(sigma=1\)

Así que si denotamos por \(Y=\sum_{i=1}^{n}\) , resulta que \(Y\) es la
suma de normales estándar \(N(\mu=0,\sigma=1)\), idénticamente
distribuidas y por lo tanto sabemos que \(Y\) sigue una ley
\(N(n\cdot 0=0, n\cdot \sqrt\cdot \sigma=\sqrt{10})\). Ahora podemos
operar

\(P\left(1.56< \sum_{i=1}^{n}< 18.31\right)=P(2.56< Y < 18.31)=P(Y< 18.31)-P(Y< 2.56)=0.9664497- 0.6010246= 0.3654251.\)

\begin{Shaded}
\begin{Highlighting}[]
\KeywordTok{pnorm}\NormalTok{(}\FloatTok{18.31}\NormalTok{,}\DataTypeTok{mean=}\DecValTok{0}\NormalTok{,}\DataTypeTok{sd=}\KeywordTok{sqrt}\NormalTok{(}\DecValTok{10}\NormalTok{))}
\end{Highlighting}
\end{Shaded}

\begin{verbatim}
## [1] 1
\end{verbatim}

\begin{Shaded}
\begin{Highlighting}[]
\KeywordTok{pnorm}\NormalTok{(}\FloatTok{2.56}\NormalTok{,}\DataTypeTok{mean=}\DecValTok{0}\NormalTok{,}\DataTypeTok{sd=}\KeywordTok{sqrt}\NormalTok{(}\DecValTok{10}\NormalTok{))}
\end{Highlighting}
\end{Shaded}

\begin{verbatim}
## [1] 0.7908986
\end{verbatim}

\begin{Shaded}
\begin{Highlighting}[]
\KeywordTok{pnorm}\NormalTok{(}\FloatTok{18.31}\NormalTok{,}\DataTypeTok{mean=}\DecValTok{0}\NormalTok{,}\DataTypeTok{sd=}\KeywordTok{sqrt}\NormalTok{(}\DecValTok{10}\NormalTok{))}\OperatorTok{-}\KeywordTok{pnorm}\NormalTok{(}\FloatTok{2.56}\NormalTok{,}\DataTypeTok{mean=}\DecValTok{0}\NormalTok{,}\DataTypeTok{sd=}\KeywordTok{sqrt}\NormalTok{(}\DecValTok{10}\NormalTok{))}
\end{Highlighting}
\end{Shaded}

\begin{verbatim}
## [1] 0.2091014
\end{verbatim}

o también, tipificando \(Z=\frac{Y}{\sqrt{10}}\) es una \(N(0,1)\)

\begin{Shaded}
\begin{Highlighting}[]
\KeywordTok{pnorm}\NormalTok{(}\FloatTok{18.31}\OperatorTok{/}\KeywordTok{sqrt}\NormalTok{(}\DecValTok{10}\NormalTok{),}\DataTypeTok{mean=}\DecValTok{0}\NormalTok{,}\DataTypeTok{sd=}\DecValTok{1}\NormalTok{)}
\end{Highlighting}
\end{Shaded}

\begin{verbatim}
## [1] 1
\end{verbatim}

\begin{Shaded}
\begin{Highlighting}[]
\KeywordTok{pnorm}\NormalTok{(}\FloatTok{2.56}\OperatorTok{/}\KeywordTok{sqrt}\NormalTok{(}\DecValTok{10}\NormalTok{),}\DataTypeTok{mean=}\DecValTok{0}\NormalTok{,}\DataTypeTok{sd=}\DecValTok{1}\NormalTok{)}
\end{Highlighting}
\end{Shaded}

\begin{verbatim}
## [1] 0.7908986
\end{verbatim}

\begin{Shaded}
\begin{Highlighting}[]
\KeywordTok{pnorm}\NormalTok{(}\FloatTok{18.31}\OperatorTok{/}\KeywordTok{sqrt}\NormalTok{(}\DecValTok{10}\NormalTok{),}\DataTypeTok{mean=}\DecValTok{0}\NormalTok{,}\DataTypeTok{sd=}\DecValTok{1}\NormalTok{)}\OperatorTok{-}\KeywordTok{pnorm}\NormalTok{(}\FloatTok{2.56}\OperatorTok{/}\KeywordTok{sqrt}\NormalTok{(}\DecValTok{10}\NormalTok{),}\DataTypeTok{mean=}\DecValTok{0}\NormalTok{,}\DataTypeTok{sd=}\DecValTok{1}\NormalTok{)}
\end{Highlighting}
\end{Shaded}

\begin{verbatim}
## [1] 0.2091014
\end{verbatim}

obtenemos el mismo resultado.

\hypertarget{soluciuxf3n-ejercicio-8}{%
\subsection{Solución ejercicio 8}\label{soluciuxf3n-ejercicio-8}}

Consideremos la muestra aleatoria simple \(X_1,X_2,\ldots,X_{n}\)de
tamaño \(n=10\) de una v.a \(X\) normal \(N(\mu=2,\sigma=4)\). Definimos
la siguiente variable aleatoria
\(Y=\frac{\sum\limits_{i=1}^{10}{(X_i-2)}^2}{16}\). Calculad
\(P(Y\leq 2.6)\)

Notemos que \(Z_i=\frac{X_i-2}{4}\) son variables \(N(0,1)\) para
\(i=1.2,\ldots,10\)

Ahora \[
Y=\frac{\sum\limits_{i=1}^{10}{(X_i-2)}^2}{16}=
\sum\limits_{i=1}^{10}\left(\frac{X_i-2}{4}\right)^2=
\sum\limits_{i=1}^{10} Z_i^2=\chi_{10}^2.
\]

Luego \(Y=\chi_{10}^2\) es una v.a. \(\chi^2\) con 10 grados de
libertad. Ya podemos calcular la probabilidad pedida
\(P(Y\leq 2.6)=P(\chi_{10}^2\leq 2.6)=0.010663.\)

El cálculo lo hemos hecho con

\begin{Shaded}
\begin{Highlighting}[]
\KeywordTok{pchisq}\NormalTok{(}\FloatTok{2.6}\NormalTok{,}\DataTypeTok{df=}\DecValTok{10}\NormalTok{)}
\end{Highlighting}
\end{Shaded}

\begin{verbatim}
## [1] 0.01066303
\end{verbatim}

\end{document}
