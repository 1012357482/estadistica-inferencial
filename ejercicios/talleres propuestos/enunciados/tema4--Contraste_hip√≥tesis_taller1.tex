% Options for packages loaded elsewhere
\PassOptionsToPackage{unicode}{hyperref}
\PassOptionsToPackage{hyphens}{url}
\PassOptionsToPackage{dvipsnames,svgnames*,x11names*}{xcolor}
%
\documentclass[
]{article}
\usepackage{lmodern}
\usepackage{amssymb,amsmath}
\usepackage{ifxetex,ifluatex}
\ifnum 0\ifxetex 1\fi\ifluatex 1\fi=0 % if pdftex
  \usepackage[T1]{fontenc}
  \usepackage[utf8]{inputenc}
  \usepackage{textcomp} % provide euro and other symbols
\else % if luatex or xetex
  \usepackage{unicode-math}
  \defaultfontfeatures{Scale=MatchLowercase}
  \defaultfontfeatures[\rmfamily]{Ligatures=TeX,Scale=1}
\fi
% Use upquote if available, for straight quotes in verbatim environments
\IfFileExists{upquote.sty}{\usepackage{upquote}}{}
\IfFileExists{microtype.sty}{% use microtype if available
  \usepackage[]{microtype}
  \UseMicrotypeSet[protrusion]{basicmath} % disable protrusion for tt fonts
}{}
\makeatletter
\@ifundefined{KOMAClassName}{% if non-KOMA class
  \IfFileExists{parskip.sty}{%
    \usepackage{parskip}
  }{% else
    \setlength{\parindent}{0pt}
    \setlength{\parskip}{6pt plus 2pt minus 1pt}}
}{% if KOMA class
  \KOMAoptions{parskip=half}}
\makeatother
\usepackage{xcolor}
\IfFileExists{xurl.sty}{\usepackage{xurl}}{} % add URL line breaks if available
\IfFileExists{bookmark.sty}{\usepackage{bookmark}}{\usepackage{hyperref}}
\hypersetup{
  pdftitle={Ejercicios Tema 4 - Contraste hipótesis. Taller 1},
  pdfauthor={Ricardo Alberich, Juan Gabriel Gomila y Arnau Mir},
  colorlinks=true,
  linkcolor=red,
  filecolor=Maroon,
  citecolor=blue,
  urlcolor=blue,
  pdfcreator={LaTeX via pandoc}}
\urlstyle{same} % disable monospaced font for URLs
\usepackage[margin=1in]{geometry}
\usepackage{graphicx,grffile}
\makeatletter
\def\maxwidth{\ifdim\Gin@nat@width>\linewidth\linewidth\else\Gin@nat@width\fi}
\def\maxheight{\ifdim\Gin@nat@height>\textheight\textheight\else\Gin@nat@height\fi}
\makeatother
% Scale images if necessary, so that they will not overflow the page
% margins by default, and it is still possible to overwrite the defaults
% using explicit options in \includegraphics[width, height, ...]{}
\setkeys{Gin}{width=\maxwidth,height=\maxheight,keepaspectratio}
% Set default figure placement to htbp
\makeatletter
\def\fps@figure{htbp}
\makeatother
\setlength{\emergencystretch}{3em} % prevent overfull lines
\providecommand{\tightlist}{%
  \setlength{\itemsep}{0pt}\setlength{\parskip}{0pt}}
\setcounter{secnumdepth}{5}
\renewcommand{\contentsname}{Contenidos}

\title{Ejercicios Tema 4 - Contraste hipótesis. Taller 1}
\author{Ricardo Alberich, Juan Gabriel Gomila y Arnau Mir}
\date{Curso completo de estadística inferencial con R y Python}

\begin{document}
\maketitle

{
\hypersetup{linkcolor=blue}
\setcounter{tocdepth}{2}
\tableofcontents
}
\hypertarget{contraste-hipuxf3tesis-taller-1.}{%
\section{Contraste hipótesis taller
1.}\label{contraste-hipuxf3tesis-taller-1.}}

Los siguientes ejercicios son de puro cálculo. Seguid la teoría y
utilizar R para el cálculo de los estadísticos y de los cuantiles de los
\(p\)-valores.

\hypertarget{ejercicio-1}{%
\subsection{Ejercicio 1}\label{ejercicio-1}}

En muestra aleatoria simple de tamaño \(n=36\) extraída de una población
normal con \(\sigma^2=12^2\) hemos obtenido la siguiente media muestral
\(\overline{x}=62.5\), Contrastar al nivel de significación
\(\alpha=0.05\), la hipótesis nula \(\mu=61\) contra la alternativa
\(\mu<60\). Resolver calculando el \(p\)-valor del contraste.

\hypertarget{ejercicio-2}{%
\subsection{Ejercicio 2}\label{ejercicio-2}}

Hemos obtenido una media muestral de \(\overline{x}=72.5\) de una
muestra aleatoria simple de tamaño \(n=100\) extraída de una población
normal con \(\sigma^2=30^2\). Contrastar al nivel de significación
\(\alpha=0.10\), la hipótesis nula \(\mu=77\) contra las siguientes tres
alternativas \(\mu\not= 70\), \(\mu>70\), \(\mu<70\). Calcular el
\(p\)-valor en cada caso.

\hypertarget{ejercicio-3}{%
\subsection{Ejercicio 3}\label{ejercicio-3}}

En un contraste bilateral, con \(\alpha=0.01\), ¿para qué valores de
\(\overline{X}\) rechazaríamos la hipótesis nula \(H_{0}:\mu=70\), a
partir de una muestra aleatoria simple de tamaño \(n=64\) extraída de
una población normal con \(\sigma^2=16^2\)?

\hypertarget{ejercicio-4}{%
\subsection{Ejercicio 4}\label{ejercicio-4}}

El salario anual medio de una muestra de tamaño \(n= 1600\) personas,
elegidas aleatoria e independientemente de cierta población de
profesionales de las Tecnologías de la Información y Comunicación (TIC)
ha sido de de 45000 euros, supongamos que nos dicen que la desviación
típica es \(\sigma=2000\) euros

\begin{enumerate}
\def\labelenumi{\arabic{enumi}.}
\tightlist
\item
  ¿Es compatible con este resultado la hipótesis nula,
  \(H_{0}:\mu=43500\) contra la alternativa bilateral, al nivel de
  significación \(\alpha=0.01\)?
\item
  ¿Cuál es el intervalo de confianza para \(\mu\)?
\item
  Calcular el \(p\)-valor del contraste.
\end{enumerate}

\hypertarget{ejercicio-5-extra-voluntario}{%
\subsection{Ejercicio 5 EXTRA
VOLUNTARIO}\label{ejercicio-5-extra-voluntario}}

Con los datos del ejercicio anterior, ¿hay evidencia sobre para oponerse
la hipótesis nula en los siguientes casos

\begin{enumerate}
\def\labelenumi{\arabic{enumi}.}
\tightlist
\item
  \(\left\{\begin{array}{ll} H_{0}:\mu=44000\\ H_{1}:\mu>44000\end{array}\right.\)
\item
  \(\left\{\begin{array}{ll} H_{0}:\mu=46250\\ H_{1}:\mu>46250\end{array}\right.\)
\end{enumerate}

\hypertarget{ejercicio-6-extra-voluntario}{%
\subsection{Ejercicio 6 EXTRA
VOLUNTARIO}\label{ejercicio-6-extra-voluntario}}

El peso medio de los paquetes de mate puestos a la venta por la casa
comercial MATEASA es supuestamente de 1 Kg. Para comprobar esta
suposición, elegimos una muestra aleatoria simple de 100 paquetes y
encontramos que su peso medio es de 0.978 Kg. y su desviación típica
\(s=0.10\) kg. Siendo \(\alpha=0.05\) ¿es compatible este resultado con
la hipótesis nula \(H_{0}:\mu=1\) frente a \(H_{1}:\mu\not=1\)? ¿Lo es
frente a \(H_{1}:\mu>1\)? Calcular el \(p\)-valor.

\end{document}
